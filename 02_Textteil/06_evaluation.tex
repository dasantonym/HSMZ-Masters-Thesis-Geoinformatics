\chapter{Discussion}
\label{ch:discussion}


\section{Application performance}
\label{sec:statistics}

The statistical evaluation focused both on the extrinsic properties of the application and the development process itself (e.g.\ code volume, complexity, time spent on development), as well as the intrinsic functional properties of the core functionality that are reflected in the cost of \ac{CPU}-, network-usage and message latency.
The cost of rendering audio and video, as well as general user experience metrics, were beyond the scope of the study, as these are highly specific to the task being implemented and are considered transient.

Two computers used in the performance evaluation were end-user laptops with the following specifications:

\emph{Computer A} was equipped with a dual-core Intel Core i5 \ac{CPU}, 16~\ac{GB} of \ac{RAM} and a 500~\ac{GB} \ac{SSD}.
\emph{Computer B} was equipped with an Apple M1 Max \ac{CPU} with ten cores, 64~\ac{GB} of \ac{RAM} and a 1~\ac{TB}~\ac{SSD}.

The test systems were located in a rural area 80~\acp{km} outside Mainz, used the Google Chrome browser (version 121.0.6167.85), ran macOS~13.6.4 and connected to a consumer \ac{ADSL} connection (60~\ac{Mbit} download and 15~\ac{Mbit} upload) via an 802.11ac wireless connection (\autoref{listing:speedtestClient}).

\begin{listing}[!ht]
\inputminted{text}{04_Artefakte/03_Listings/speedtest-client.txt}
\caption{Speedtest: connection statistics for the test clients\protect}
\label{listing:speedtestClient}
\end{listing}

Latency was measured by time stamping each message in the data producer and then logging the arrival time in the local and the remote browser.
Both computers synchronised their clocks using the same \ac{NTP} server on the local netork.
The synchronisation was refreshed every 10 seconds during the data sampling and provided sub-millisecond clock accuracy throughout the measuring process.
The measurements ran for ten minutes each and always used the \ac{BVH} data producer, sending the message type for movement qualities alongside 29~key points at a rate of 25~messages per second.
The payload size was 453~bytes for each message, amounting to a required bandwidth of about 11.3~\ac{KB/s} for each motion capture data stream.
They were repeated ten times to account for overall network service quality variance.
The audio streams were published alongside the data packets but were not measured for latency.
The \ac{CPU}, \ac{RAM} and bandwidth usage for LiveKit on the server was recorded as well.
The results of the latency analysis are shown in separate graphs for each computer containing an average across all datasets for the local and remote messages on each device.

\begin{figure}[!h]
\centering
\includesvg[width=\textwidth]{04_Artefakte/01_Abbildungen/latency-computer-a-average}
\caption[Message latency on Computer A]{Computer A: Average latency in milliseconds for local and remote messages\protect}
\label{fig:latencyComputerA}
\end{figure}

Results for computer A (\autoref{fig:latencyComputerA}) show an overall average latency of approx.~30~\ac{ms} for remote messages received over the \ac{WebRTC} connection and approx.~2~\ac{ms} for the connection from the local data producer to the browser.
The averages show a standard deviation of approx.~1.4~\ac{ms} for the remote connection and about approx.~0.2~\ac{ms} for the local connection.

\begin{figure}[!h]
\centering
\includesvg[width=\textwidth]{04_Artefakte/01_Abbildungen/latency-computer-b-average}
\caption[Message latency on Computer B]{Computer B: Average latency in milliseconds for local and remote messages\protect}
\label{fig:latencyComputerB}
\end{figure}

The results gathered on computer B (\autoref{fig:latencyComputerB}) also show an average latency of approx.~30~\ac{ms} for the remote messages and approx.~1.7~\ac{ms} for the local data producer connection.
Here, the standard deviation is approx.~2.1~\ac{ms} for the remote connection and approx.~0.6~\ac{ms} for the local connection.

On the server, the average overall network consumption for the LiveKit server was at approx.~106~\ac{KB/s} for incoming and approx.~156~KB/s for outgoing traffic, moderately fluctuating with a peak of approx.~122~\ac{KB/s} incoming and approx.~198~\ac{KB/s}, while the isolated packet traffic remained relatively stable at an average of approx.~32~\ac{KB/s} (in) and approx.~33~\ac{KB/s} (out) with respective maxima of approx.~35~\ac{KB/s} and approx.~36~\ac{KB/s} (\autoref{fig:serverNetworkUsage}).
The server compute resource usage was minimal with only an average of approx.~0.04~\ac{CPU} cores and approx.~81~\ac{MB}~\ac{RAM} usage (\autoref{fig:serverComputeResources}).

\begin{figure*}[!ht]
\includesvg[width=.48\textwidth]{04_Artefakte/01_Abbildungen/server-bandwidth-average}\hfill
\includesvg[width=.48\textwidth]{04_Artefakte/01_Abbildungen/server-packet_bytes-average}
\caption[Server network usage during tests]{Average network usage for the LiveKit server during tests (\ac{KB/s})\protect}
\label{fig:serverNetworkUsage}
\end{figure*}

\begin{figure*}[!ht]
\includesvg[width=.48\textwidth]{04_Artefakte/01_Abbildungen/server-cpu-average}\hfill
\includesvg[width=.48\textwidth]{04_Artefakte/01_Abbildungen/server-ram-average}
\caption[Server compute resources usage during tests]{Average \ac{CPU} and \ac{RAM} usage for the LiveKit server during tests\protect}
\label{fig:serverComputeResources}
\end{figure*}

Overall, the performance analysis provided a very positive result as even the message latency for the home network connection on a moderately equipped laptop (Computer A) remained at around 30~\ac{ms}, which is well below the lower tolerance limit of 160~\ac{ms} proposed in a study by~\cite{audioLatency} on tolerable audio latency for live music creation.
On the server side, the impact on the computing resources was barely registering, the network throughput left significant headroom for more connections over the server's a gigabit connection and would most likely work as well on a 100~\ac{Mbit} connection and a server with fewer cores than the test setup.

Another pairing of test computers was attempted using a local Ethernet connection on campus at Mainz University of Applied Sciences, but was discarded, because latencies were in the single digit to sub-milliseconds range and thus could not be properly measured due lack of a precision clock system.

\section{Workload evaluation}\label{sec:workload-evaluation}

The evaluation of time spent on the development work was based on the timesheets kept during the process.
Only the time spent on the actual programming and infrastructure creation was tracked, as the research could not be adequately separated from the work on the study itself.
All recorded tasks were categorised by the language used (e.g.\ JavaScript versus Python), the component worked on (e.g. \ac{UI} vs. \ac{API}) and the type of work (e.g.\ programming versus administration).

In total, approx.~144 work hours were spent creating the reference implementation and its deployment on the test infrastructure.
This would amount to approx.~18 workdays or three and a half five-day workweeks, assuming eight hours for each day, which would be in line with §3 of the German law for labour time regulation, known as the \textquote{Arbeitszeitgesetz} (see \parencite{abzgPar3}).

The distribution of time spent on software development versus documentation and setting up the infrastructure (administration) and constructing the head-tracker\textquotesingle s hardware implementation (\autoref{fig:timeSpentWorkareas}) clearly shows that the largest amount of time (approx.~93\%) was spent on programming and the latter two factors were only marginal in the effort needed to be put in.

\begin{figure*}[!ht]
\centering
\includesvg[scale=0.8]{04_Artefakte/01_Abbildungen/time-spent-areas-percent}
\caption[Distribution of time spent on development]{Distribution of time spent on development by work area, language and application component\protect}
\label{fig:timeSpentWorkareas}
\end{figure*}

Of the amount of time spent in software development, the time spent by programming languages (\autoref{fig:timeSpentLanguages}) shows a clear majority of approx.~72\% for \ac{JS}, given that it was chosen as the language for the \ac{UI} and the \ac{API} and the project\textquotesingle s general primary language.
However, Python still takes up approx.~21\% of the work hours spent on programming.
C++ required only a marginal amount of work with approx.~7\% of the time spent.

\begin{figure*}[!ht]
\centering
\includesvg[scale=0.8]{04_Artefakte/01_Abbildungen/time-spent-languages-percent}\hfill
\caption[Distribution of time spent on development]{Distribution of time spent on development by programming language\protect}
\label{fig:timeSpentLanguages}
\end{figure*}

Time spent on the different components (\autoref{fig:timeSpentComponents}) shows that about half of the time was spent on the user interface (approx.~56\%). The component using the second largest amount of time is the Python data producer (approx.~20\%).
All other components remained under 10\% and notably, the \ac{SDK} required only 7.3\% of time as it was mostly constructed by refactoring existing code and adding tests.
Also noteworthy is the time spent on the \ac{API} server, which was only approx.~4\% since it was mainly autogenerated and required only very litte modification.
Equally minimal is the time spent on the deployment, also with only approx.~4\%.

\begin{figure*}[!ht]
\centering
\includesvg[scale=0.8]{04_Artefakte/01_Abbildungen/time-spent-components-percent}\hfill
\caption[Distribution of time spent on development]{Distribution of time spent on development by programming language\protect}
\label{fig:timeSpentComponents}
\end{figure*}

The overall time spent of about a month should be seen in the context of an ongoing larger project and on the basis of parts of the application being meant for reuse.
If the \ac{API} and data producers would be kept largely stable and only the \ac{UI} would be subject for modifications, extensions or even complete replacement, this would affect roughly half of time spent on the initial development.
While an accurate prediction on the time spent on subsequent modifications cannot be made, it should equally stay under a month and thus be an approprate time investment to support future projects.

\section{Code quality}
\label{sec:code-quality}

\begin{figure}[!ht]
\centering
\includesvg[scale=0.5]{04_Artefakte/01_Abbildungen/sonarcloud-cognitive_complexity}
\caption[Cognitive complexity]{Cognitive complexity per component\protect}
\label{fig:cognitiveComplexity}
\end{figure}

Code quality was analysed using the service SonarCloud\footnote{\url{https://sonarcloud.io}}, which is free for small projects and provides an extensive set of metrics, of which the code weight (files, lines of code) and complexity were of specific interest, as defined in \autoref{ch:methodology}.
The default recommended limit for cognitive complexity set by the SonarCloud code analysis setup is 15 for \ac{JS} and Python and 25 for C++.
The cognitive complexity (\ref{fig:cognitiveComplexity}) was only calculated for the core \ac{SDK}, the \ac{UI}, \ac{API} and the general data producer.
The Arduino code does not use any branching, and the C++ code for the Captury producer was deemed a workaround since it should be integrated into the Python structure for proper use.
The maximum complexity shows a significant overhead exceeding the recommended thresholds (see \autoref{ch:methodology}) on behalf of the core \ac{SDK} with 15 and for the data producer with 19, although the median values show a distinct concentration of complexity with the data producer (6 versus 1 for the \ac{SDK}).
This shows that the massive complexity value of 15 is an outlier for the \ac{SDK} but seems more intrinsic to the \textquote{DataProducer} class structure.
However, the \ac{UI} and \ac{API} show a very low overall complexity, which was the desired outcome to keep these parts more hackable and easy to grasp.

In terms of weight measured by the number of files, the most code files were produced for the \ac{UI} and the \ac{API}, which aligns with the distribution of functionality according to the concept.
The distribution of weight regarding the total lines of code without comments produced for each component aligns with the number of files for the \ac{UI} but shows a significant overhead for the data producer with almost as many lines of code distributed among a third of the number of files (\autoref{fig:codeWeight}).
It generally shows a relatively moderate weight for the entire application, with a maximum of 34 files created for the \ac{UI} and a total of 83 files across all components.

\begin{figure*}[!ht]
\includesvg[width=.48\textwidth]{04_Artefakte/01_Abbildungen/sonarcloud-filecount}\hfill
\includesvg[width=.48\textwidth]{04_Artefakte/01_Abbildungen/sonarcloud-ncloc_sum}
\caption[Source file count and total lines of code]{Number of source files and total lines of code per application component\protect}
\label{fig:codeWeight}
\end{figure*}

Median and maximum amounts of lines of code across files for each component (\autoref{fig:linesOfCode}) show a moderate distribution with the median at about half or a third of the maximum value and below the recommended thresholds (see \autoref{ch:methodology}).
Again, the only exception is the data producer that, while showing a moderate median value, shows a maximum of 424 lines of code.

\begin{figure}[!ht]
\centering
\includesvg[scale=0.5]{04_Artefakte/01_Abbildungen/sonarcloud-ncloc_files}
\caption[Lines of code]{Maximum and median number of lines of code per application component\protect}
\label{fig:linesOfCode}
\end{figure}

Summarily, the results for the code quality of such a somewhat non-standard development process provide further encouragement to pursue more projects based on the same strategy or even on the study\textquotesingle s resulting source code.
The average complexity and weight of the code, along with the documentation and comments added, should enable third-party developers to engage with the results by reviewing and modifying specific parts to introduce their own experimental data capture, visualisation or sonification methods.

\section{Critical reflection}
\label{sec:critical-reflection}

The development process followed the guiding principles defined in the application concept (see \autoref{ch:concept}).
It was a pleasant experience with the selected frameworks delivering on their promised functionality and ease of use.
The initial setup was quick and simple due to the ease of setup of the LiveKit \ac{WebRTC} server and the generation of boilerplate code for the \ac{API} and the \ac{UI}.
A development environment was set up equally quickly, and work immediately started with practical experimentation, providing a motivational boost by quickly establishing tangible results.
Web standards integration was simple and efficient through directly implemented standards such as WebSockets and WebBluetooth and the other ones integrated through third-party libraries.
However, the implementation of the web audio standard still leaves a lot to desire, especially the support for customised spatial audio in the browser.
Currently, there is no built-in way to load custom \ac{HRTF} data, which would drastically improve the accuracy of spatial positioning for sound.
There are approaches using a custom build of the Chromium browser~\parencite{chromiumCustomHrtf} or a custom audio node~\parencite{customHrtfAudioNode}, which unfortunately does not work with the \ac{SOFA} file format, and due to the study's time constraints did not make it into the reference implementation.

The selection of appropriate tools and libraries might be a challenge for many beginners with basic programming knowledge who are not primarily working in web development.
As the libraries and frameworks that make up the broad spectrum of available web development technologies tend to favour different paradigms that, in turn, are also subject to frequent change due to trends and \textquote{hype cycles}, it can be challenging to keep up and daunting to make an informed choice among the available options.
Looking at the options evaluated throughout this study, there are a few factors to consider when deciding which framework to prefer for a specific type of project.
It is vital to decide if a project will be maintained in the long run.
If, as in this case, the implementation is transient and does not require long-term maintenance, then the choice should be guided by the provided feature set and the paradigms implemented.
While it is always a good choice to use a library with a large community and a longstanding presence, a more experimental and niche case might require something newer or less popular.

For the development process itself, the strategy of getting something up and running as quickly as possible, working on that while regularly refactoring and restructuring, was a gratifying experience.
While this approach might be perceived as not exactly team-friendly due to the danger of conflicting work, it can still be pursued in very small teams by agreeing on basic protocols and then working on separate components.
The partitioning of the application into several components was also highly beneficial because it allowed for different languages to be used where appropriate.
There are many more implementations of movement analysis for Python than for JavaScript or Node.js.
Before porting functionality to a different language, it is easier to set up a separate microservice or tool and use a standardised messaging protocol to communicate with the rest of the application.
Another benefit of the partitioning design pattern is that components can be discarded and replaced with different technologies and supporting libraries without necessarily influencing the application\textquotesingle s overall functionality.
It is important to note, though, that while the \ac{UI} is thought of as transient and unstable, unit testing is still highly recommended for the stable components that should keep being used outside of the specific implementation.
While it could be beneficial to develop the core \ac{SDK} using test-driven development and, once stable, working on the implementation, it would add a period during which there would be only theoretical planning and development.
This seems out of place for small creative projects that work interdisciplinary, like a dance production that needs its own remote collaboration tool and that needs to get started as quickly as possible.
Nonetheless, it is still important to have a finalising phase in which the core functionality is extracted, documented and outfitted with tests to keep the \ac{SDK} as an artefact on which to base further projects.
