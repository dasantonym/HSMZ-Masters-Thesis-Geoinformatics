\chapter{Conclusion and outlook}
\label{ch:conclusion-and-outlook}

The survey of available technologies and methods in \autoref{ch:conceptualfoundations} showed a broad range of readily available options.
Based on the methodology established in \autoref{ch:methodology}, a reference implementation could be created (\autoref{ch:concept} and \autoref{ch:implementation}) and its discussion in \autoref{ch:discussion} produced a positive recommendation for such a \textquote{single-use} or \textquote{ephemeral} development.
This seems especially valuable in cases without a focus on commercial deployment and exploitation of services or products but rather specialised tools that are an intrinsic part of precariously funded, smaller, self-contained and short-lived projects.
Additionally, the resulting reusable components accelerate future development if applied roughly in the same area of usage or recombined with other new additions.

Regarding the bulk of the application as transient and extracting the core functionality into well-documented and tested modules further has the benefit that if the application is passed on to other developers for another project or task, they can decide to either go with the existing base infrastructure or to take only the core functionality and to implement it in their favoured environment.
It enables them to port the core to other languages more easily.
In web development, this is essential because, while the standards and the application\textquotesingle s feature set might stay the same, the framework and tooling landscape certainly doesn\textquotesingle t, and even just a few years can render the application obsolete if it is not constantly maintained.

Funding schemes for most digital projects in niche culture or arts disciplines usually do not allocate sums that would allow for more than one or two developers to be included on the team.
Often, the people working on these projects are not formally trained software engineers but rather creative coders, hackers or engaging from a multidisciplinary angle and for whom the technological aspects came as a secondary interest to a primary education in the performing arts.
However, there is a growing focus on exploring digital methods in various disciplines, such as the digital humanities and, depending on the institution, in the performing or fine arts.
This suggests approaching software development differently and valuing the development process over the actual result.
While the objective is always to produce a functional implementation, it is less critical to insist on clean code or engineering virtuosity such as scalability, maintainability, and perfect abstraction.
It aims to \emph{decommodify the notion of an application} and to turn it into an ephemeral \emph{statement within a dialogue between engineering and artistic practice}.

We propose the expression of \emph{code composting} to describe a cyclical development process that alternates between intuitive composition focused on loose experimentation and analytical decomposition using reflection and refactoring to extract essential structures emergent in the compositional process.
The extracted components then become a \emph{substrate} for subsequent development processes, repeating the cycle and producing additional building blocks.
Here, the focus shifts away from traditional coding virtuosity and can help to lower the barrier between engineers and artists.
This should enable a more intertwined and participatory dialogue between all project\textquotesingle s participants that is not just expressed in development meetings but actual proposed hacks and modifications or additions, given the interest on behalf of all parties involved.

Code composting values \emph{processes over products} and embraces the \emph{ephemerality of contemporary digital technologies} as intrinsic design factors.
This method is explicitly recommended for experimental, non-commercial endeavours where factors like scale or long-term maintainability are neglectable and where development should happen from within and be shaped by artistic practice instead of trying to abstract and delegate this process to external service providers.
Thus, development should focus on constant reflection and re-evaluation of development results and be ready to potentially discard everything but a seed for a fresh start while keeping the knowledge and insights gained in the process alive through continuous inclusive, transdisciplinary dialogue and experimentation.
It positively values what otherwise would be negatively termed as \textquote{rot}, resulting from a messy growth process in an amalgamation of engineering and artistic experimentation.
Instead of a solutionist approach that arranges everything around analytical engineering, working from a supposedly well-defined \textquote{problem} reduced to mathematical abstractions, it assumes a position within a constant state of ambivalence, uncertainty and flux.
Drawing on a messy plurality and richness that exists in the interdisciplinary, interpersonal and intuitive, we acknowledge that \textquote{we are humus, not Homo, not anthropos; we are compost, not posthuman.}\parencite[][55]{harawayStayingWithTheTrouble}
