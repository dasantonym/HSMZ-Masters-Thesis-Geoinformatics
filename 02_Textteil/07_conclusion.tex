\chapter{Conclusion}

The result of this feasibility study supports the recommendation for \textquote{single-use} or rather \textquote{ephemeral} or \textquote{transient} development in cases where there is not a focus on commercial deployment of services or products, but rather specialised tools that are an intrinsic part of smaller self-contained or short-lived projects. In this case, a full-time developer could get the application up and running in about three weeks, not accounting for the value of reusable components for later iterations.

Regarding the bulk of the application as transient and extracting the core functionality into well-documented and tested modules also has the benefit, that if the application is passed on to other developers for another project or task, they can decide to either go with the existing infrastructure or to take only the core and implementing it in their own favoured environment. It would even enable them to more easily port the core to other languages. In the web development area, this is essential, because, while the standards and the application's feature set might stay the same, but the framework and tooling landscape most certainly doesn't and even just a few years can render the application obsolete, if it is not constantly maintained.

As this study was conducted by a single developer, the aspect of team dynamics is not accounted for. The unstable nature of the initial development process might pose a problem for development teams, because even if the work is strictly divided among the various team-members, it might still lead to conflicts when bringing the disparate application parts together. However, this could be alleviated by formulating strict message formats and protocols beforehand and then assigning separate components to developers that communicate via these protocols. Additionally, the reality for most digital projects in niche culture or arts disciplines usually do not get funding allocations that would allow for more than one or two developers to be included on the team.

In general, this method of development can explicitly be recommended for experimental, non-commercial endeavours where factors like scale or long-term maintainability are secondary and where development should happen from within the actual practice as opposed to trying to communicate wishes and needs to external service providers. It aims to \textquote{decommodify} the notion of an application and to turn it into an ephemeral statement made within an artistic dialogue.

This results in \textquote{compostable software} that grows and eventually decays but leaves its essential functionality in the form of small modules or even just documented methods and thought processes that can be used to support future development processes. Additionally, even if the application is meant to be discarded in its entirety after the project, the focus no longer needs to be on coding virtuosity and can help to lower the barrier between engineer and artist. This should enable a more intertwined and participatory dialogue between the project's participants that is not just expressed in development meetings, but actual proposed hacks and modifications or additions, given the interest on behalf of all parties involved.

Compostable software development is about valuing processes over products and embracing the ephemerality of digital technological techniques as intrinsic design factors. Thus, it should focus on constant reflection and re-evaluation of development results and being ready to potentially discarding everything but a seed for a fresh start, but keeping the knowledge and insights gained in the process alive through continuous inclusive, transdisciplinary dialogue and experimentation. It positively values, what otherwise would negatively be described as \textquote{rot}, the result of a messy growth process in an amalgamation of engineering and artistic experimentation.
