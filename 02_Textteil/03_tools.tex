\chapter{Tools}



\section{Web Development Languages}

\subsection{JavaScript}

\subsection{TypeScript}


\section{Native Application Development}

\subsection{Node JS}

\subsection{Python}

\subsection{C/C++}



\section{Frontend Frameworks and Libraries}

There is a wide range of available \ac{JS} frameworks to build dynamic frontends for \ac{SPA}s and \ac{PWA}s. The three libraries currently dominating the landscape are \emph{React}, developed by \emph{Facebook} in 2013, and \emph{Vue.js}, developed by Evan You in 2014. These libraries can be used in conjunction with frameworks to offer complete solutions providing routing and state management. Another popular framework is \emph{Angular}, which was originally released by \emph{Google} in 2010, then re-released in 2016.

\begin{figure}[h]
    \centering
    \includegraphics[scale=0.4]{04_Artefakte/01_Abbildungen/stateofjs-usage-frontend-frameworks-2022}
    \caption[{Most used frontend frameworks in 2022}]{State of JS: Most used frontend frameworks in 2022\protect\footnotemark}
    \label{fig:mostUsedFrameworks}
\end{figure}
\footnotetext{\cite{mostUsedFrontendFrameworks22}}

\subsection{React}

\emph{React} (\url{https://react.dev/}), developed by \emph{Facebook} and maintained by its successor \emph{Meta}, has become the most widely used tool for building \ac{SPA}s and is steadily leading the rankings for most used frontend frameworks both in the \emph{StackOverflow} \parencite{stackOverflowPollWebFrameworks23} and the \emph{State Of JS} \parencite{mostUsedFrontendFrameworks22} polls. By definition it is not a framework, but a \ac{UI} library that builds on other extensions to support state-management, routing and deployment functionality. Although it is not a framework itself, there are existing frameworks like \emph{Next.js} (\url{https://nextjs.org/}) for the web and \emph{ReactNative} (\url{https://reactnative.dev/}) for building mobile apps using native functionality. React makes use of \ac{JSX} which allows directly mixing inline \ac{HTML} with the \ac{JS} or \ac{TS} code structure.

\subsection{Vue.js}

\emph{Vue.js} (\url{https://vuejs.org/}) was developed by Evan You and is maintained by an international team of individuals. In the first years after its inception it had a rather marginal presence. This can be at least partially attributed to the fact that it originated in China and most of its supporting modules were localised in Chinese language. Over the years it grew in popularity and received much more international support, eventually overcoming the language barrier. Unlike \emph{React}, it is billed as a "progressive framework" that provides very basic functionality for building reactive components, but also accommodates more complex use-cases \parencite{vueProgressiveFramework}. \emph{Vue.js} builds on standard \ac{JS} or \ac{TS}, \ac{HTML} and \ac{CSS} to build components, recommending a simple template mechanism mixed with reactive substitutions. However, it also supports using \ac{JSX} for specifying inline \ac{HTML} within \ac{JS}. As with \emph{React}, there are extensions and frameworks like \emph{Quasar} (\url{https://quasar.dev/}) and \emph{Nuxt} (\url{https://nuxt.com/}) that enable even more sophisticated workflows application development and deployment.

\subsection{Angular}

\emph{Angular} was initially released by \emph{Google} in 2010 as \emph{AngularJS}  and officially discontinued in 2022 (\url{https://angularjs.org/}). A completely overhauled and currently used version 2 was then released in 2016 and is maintained by \emph{Google}. It is different from \emph{React} and \emph{Vue.js} in that it is a complete framework that contains everything required to build and deploy an application and it explicitly recommends \ac{TS} as a programming language. The framework also is less flexible in that it is opinionated and has its own set of best practices baked into the framework's structure.




\section{Backend Libraries}

\subsection{Express}

\subsection{Koa}

\subsection{Fastify}


\section{Backend Frameworks}

\subsection{Nest JS}

\subsection{Meteor}

\subsection{Feathers}

