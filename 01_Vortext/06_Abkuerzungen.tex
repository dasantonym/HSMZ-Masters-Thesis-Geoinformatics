% Erklärender Text zu dieser Datei --------------------------------------------------------
% Die Datei 06_Abkuerzungen.tex dient der Definition von Abkürzungen, die im Abkürzungsverzeichnis gelistet werden sollen.
% Zur besseren Übersichtlichkeit wird empfohlen die Abkürzungen zentral in dieser Datei zu definieren.
%
% Abkürzungen müssen erst definiert werden mittels \acro bevor diese im eigentlichen Textteil mit \ac verwendet werden können.
% Die Angabe von Übersetzungen zu Abkürzungen ist möglich mittels der Anweisung \acroforeign.
% -----------------------------------------------------------------------------------------
\begin{acronym}
	\acro{2D}{2-dimensional}
	\acro{3D}{3-dimensional}
	\acro{ADSL}{asynchronous digital subscriber line}
	\acro{AJAX}{asynchronous JavaScript and XML}
	\acro{API}{application programming interface}
	\acro{AR}{augmented reality}
	\acro{BVH}{Biovision hierarchy}
	\acro{CGI}{computer-generated imagery}
	\acro{CLI}{command-line interface}
	\acro{CPU}{central processing unit}
	\acro{CNCF}{Cloud Native Computing Foundation}
	\acro{CRUD}{create, retrieve, update and delete}
	\acro{CSS}{Cascading Style Sheets}
	\acro{DIY}{do-it-yourself}
	\acro{ES6}{ECMAScript 6}
	\acro{GATT}{Generic Attribute Profile}
	\acro{GB}{gigabyte}
	\acro{GIS}{geographic information system}
	\acro{HRTF}{head-related transfer function}
	\acro{HTML}{HyperText markup language}
	\acro{HTTP}{HyperText transmission protocol}
	\acro{I/O}{input/output}
	\acro{I2C}{Inter-Integrated Circuit}
	\acro{IETF}{Internet Engineering Task Force}
	\acro{IMU}{inertial measurement unit}
	\acro{JGU}{Johannes Gutenberg-University Mainz}
	\acro{JS}{JavaScript}
	\acro{JSON}{JavaScript Object Notation}
	\acro{JSX}{JavaScript XML}
	\acro{KB}{kilobyte}
	\acro{KB/s}{kilobyte per second}
	\acro{km}{kilometre}
	\acro{LE}{little-endian}
	\acro{MB}{megabyte}
	\acro{Mbit}{megabit}
	\acro{MCU}{multipoint control unit}
	\acro{ML}{machine learning}
	\acro{ms}{millisecond}
	\acro{NoSQL}{not-only SQL}
	\acro{NPM}{Node Package Manager}
	\acro{NTP}{Network Time Protocol}
	\acro{OCI}{Open Container Initiative}
	\acro{OOP}{object-oriented programming}
	\acro{OS}{operating system}
	\acro{P2P}{peer-to-peer}
	\acro{PC}{personal computer}
	\acro{PWA}{progressive web application}
	\acro{RAM}{random access memory}
	\acro{RFC}{request for comments}
	\acro{RTC}{real-time communication}
	\acro{SATC}{Software Assurance Technology Center}
	\acro{SDK}{software development kit}
	\acro{SFU}{selective forwarding unit}
	\acro{SOFA}{spatially oriented format for acoustics}
	\acro{SPA}{single-page application}
	\acro{SSD}{solid-state drive}
	\acro{SSL}{secure sockets layer}
	\acro{SQL}{structured query language}
	\acro{TCP}{transfer control protocol}
	\acro{TB}{terabyte}
	\acro{TS}{TypeScript}
	\acro{UDP}{user datagram protocol}
	\acro{UI}{user interface}
	\acro{UML}{unified modeling language}
	\acro{VPN}{virtual private network}
	\acro{VR}{virtual reality}
	\acro{W3C}{World Wide Web Consortium}
	\acro{WebRTC}{web real-time communication}
	\acro{WebXR}{web mixed reality}
	\acro{WHATWG}{Web Hypertext Application Technology Working Group}
	\acro{XML}{extensible markup language}
	\acro{XR}{mixed reality}
\end{acronym}
