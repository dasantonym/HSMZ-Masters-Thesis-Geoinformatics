\pdfbookmark{Abstract}{abstract}
\section*{Abstract}

This study investigates the feasibility of creating a fully customised and task-specific telepresence application based exclusively on open web standards and free software for contemporary dance practice.
It surveys existing technologies and paradigms and establishes a reference implementation called \textquote{Sensorama} to evaluate its abstract functionality and the development process.
The existing technological landscape and the feasibility of producing task-specific web applications as an intrinsic component of interdisciplinary projects focusing on digital practice are positively assessed.
A development strategy termed \textquote{code composting} is deduced from critical reflection and recommended for smaller creative projects, describing cycles of intuitive composition and analytical decomposition.

\textbf{Keywords:} telepresence, contemporary dance, motion capture, arts, culture, open-source, computer science, software engineering


\begin{otherlanguage}{ngerman}
\section*{Kurzzusammenfassung}
Diese Studie untersucht die Machbarkeit der Entwicklung einer aufgabenspezifischen Telepräsenzanwendung für den Einsatz im Zeitgenössischen Tanz und basierend auf offenen Standards.
Es wird ein Überblick über existente Technologien erabeitet und eine Referenzimplementierung mit dem Titel \glqq Sensorama\grqq \ erstellt, deren Entwicklungsprozess und abstrakte Funktion evaluiert werden.
Die Praktikabilität und technologischen Möglichkeiten einer fallspezifischen Softwareimplementierung als Teil interdisziplinärer Projekte werden positiv bewertet.
Daraus wird der Begriff der \glqq Code-Kompostierung\grqq \ abgeleitet, der sich als Entwicklungsstrategie auf Zyklen intuitiver Komposition und analytischer Dekomposition bezieht und für kleinere kreative Projekte empfohlen wird.

\textbf{Schlagwörter:} Open-Source, Telepräsenz, Zeitgenössischer Tanz, Bewegungserfassung, Kunst, Kultur, Informatik, Softwareentwicklung
\end{otherlanguage}
