\pdfbookmark{Abstract}{abstract}
\addchap*{Abstract}

This study investigates the feasibility of creating a fully customised and task-specific telepresence application based exclusively on open web standards and free software for use in contemporary dance practice.
It surveys existing technologies and paradigms and establishes a practical reference implementation to evaluate its basic functionality and the development process.
The study arrives at a positive assessment of the existing technological landscape and the feasibility to produce task-specific web applications as an intrinsic component in a smaller interdisciplinary projects with a strong focus on digital practice.
It concludes with a proposal for an alternative development method termed \textquote{compostable software}, describing a cyclical process of composition and decomposition.

\textbf{Keywords:} compostable software, open-source, telepresence, contemporary dance, motion capture, arts, computer science, software engineering


\section*{Kurzzusammenfassung}

Diese Studie untersucht die Machbarkeit der Entwicklung einer aufgabenspezifischen Telepräsenzanwendung für den Einsatz im zeitgenössischen Tanz und basierend auf offenen Standards.
Es wird ein Überblick über existente Technologien erabeitet und eine Referenzimplementierung erstellt, deren Entwicklungsprozess und abstrakte Funktion evaluiert wird.
Die Studie gelangt zu einer positiven Einschätzung im Hinblick auf die technologischen Möglichkeiten und die Praktikabilität einer fallspezifischen Softwareimplementierung als Teil von interdisziplinären Projekten.
Sie schlägt den Begriff der \textquote{kompostierbaren Software} vor, eine alternative Entwicklungsmethode bezeichnend, die sich auf zyklische Prozesse von Komposition und Dekomposition stützt.
 
 \textbf{Schlagwörter:} kompostierbare Software, Open-Source, Telepräsenz, Zeitgenössischer Tanz, Bewegungserfassung, Kunst, Informatik, Softwareentwicklung
 