\pdfbookmark{Abstract}{abstract}
\section*{Abstract}

This study investigates the feasibility of creating a fully customised and task-specific telepresence application based exclusively on open web standards and free software for contemporary dance practice.
It surveys existing technologies and paradigms and establishes a reference implementation called \textquote{Sensorama} to evaluate basic functionality and the development process.
The existing technological landscape and the feasibility of producing task-specific web applications as an intrinsic component of interdisciplinary projects focusing on digital practice are positively assessed.
An alternative development strategy termed \textquote{code composting} is proposed and recommended for smaller creative projects, describing a process-oriented cyclical method of intuitive composition and analytical decomposition.

\textbf{Keywords:} telepresence, contemporary dance, motion capture, arts, open-source, computer science, software engineering


\section*{Kurzzusammenfassung}

Diese Studie untersucht die Machbarkeit der Entwicklung einer aufgabenspezifischen Telepräsenzanwendung für den Einsatz im zeitgenössischen Tanz und basierend auf offenen Standards.
Es wird ein Überblick über existente Technologien erabeitet und eine Referenzimplementierung names \textquote{Sensorama} erstellt, deren Entwicklungsprozess und abstrakte Funktion evaluiert wird.
Die technologischen Möglichkeiten und die Praktikabilität einer fallspezifischen Softwareimplementierung als Teil von interdisziplinären Projekten werden positiv bewertet.
Es wird der Begriff der \textquote{Code-Kompostierung} eingeführt, der sich als prozessorientierte Entwicklungsstrategie auf Zyklen intuitiver Komposition und analytischer Dekomposition stützt und für kleinere kreative Projekte zu empfohlen wird.
 
 \textbf{Schlagwörter:} Open-Source, Telepräsenz, Zeitgenössischer Tanz, Bewegungserfassung, Kunst, Informatik, Softwareentwicklung
 