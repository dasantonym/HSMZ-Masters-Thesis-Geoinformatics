% Erklärender Text zu dieser Datei --------------------------------------------------------
% Die Datei Packages.tex dient als zentrale Datei, in der alle genutzten Packages geladen und konfiguriert werden.
% Basierend auf den Anforderungen sind ggf. Konfigurationen vereinzelt zu ändern - entsprechende Stellen sind mit %TODO gekennzeichnet.
% Beispiele für solche Anforderungen sind
%  - Package biblatex: Definition, des Zitationsstils und Aufbau des Literaturverzeichnisses; Default ist IEEE, Konfiguration für APA ist hinterlegt
%  - Package minted: Definition, wie Source Code von Sprachen dargestellt werden soll
%  - Zur korrekten Darstellung des Inhaltsverzeichnisses die "breiteste" Seitenzahl angeben
%
% HINWEIS: Die Reihenfolge, in der Packages geladen werden ist wichtig. Hinweise aus den Dokumentationen der einzelnen Packages sind daher unbedingt zur berücksichtigen, wenn weitere Packages aufgenommen werden sollen.
% -----------------------------------------------------------------------------------------
% \usepackage[a4paper, top=2.5cm]{geometry}
\usepackage[utf8]{inputenc}
\usepackage[T1]{fontenc}
\usepackage[UKenglish]{babel}
\usepackage{lmodern}
\usepackage{fancyhdr}
\usepackage{xspace} % Leerzeichen hinter parameterlosen Makros nicht als Endzeichen interpretieren
\usepackage{graphicx} %Abbildungen
\usepackage{pdfpages} %Einfügen von PDF-Seiten
\usepackage[inkscapeformat=png]{svg} %Einfügen von SVG Grafiken
\includepdfset{scale=0.7, frame, pagecommand={\thispagestyle{plain}}}
\graphicspath{{04_Artefakte/01_Abbildungen/}}
\usepackage{caption} % Bildunterschriften
\usepackage{subcaption} % https://www.ctan.org/pkg/subcaption

\usepackage{tabularx} % Tabellen
\usepackage{booktabs} % Bessere Tabellen
\usepackage[longtable]{multirow}
\usepackage{longtable}

\usepackage{amsmath}
\usepackage{amsfonts}
\usepackage{bbm}

\usepackage{xcolor}
%\usepackage{chngcntr} % fortlaufende Nummerierung von Fußnoten

\usepackage[colorinlistoftodos]{todonotes} % to disable todos use option disable, alternatively use obeyDraft or obeyFinal
\usepackage{blindtext}

\usepackage{microtype} % auskommentieren moeglich, wenn Typografie nicht zufriedenstellend

\usepackage[bottom]{footmisc} % https://golatex.de/viewtopic.php?f=21&t=24052
% Verlinkung und PDF Bookmarks https://tex.stackexchange.com/a/83051
\usepackage[nospace]{varioref}
% extensions keep all links black, only urls are blue: https://tex.stackexchange.com/a/401885/220502
\usepackage[hidelinks, colorlinks, allcolors=., urlcolor=blue]{hyperref}
\usepackage{bookmark}
\hypersetup{
    pdftitle={\titel},
    pdfauthor={\autor},
    pdfcreator={\autor},
    pdfsubject={\titel},
    pdfkeywords={\titel},
}
\usepackage{cleveref}

% Literaturverzeichnis und Quellenverwaltung mittels biblatex
%TODO Änderungen des Zitationsstils und Literaturverzeichnisses
% Zitationsstil hier auswählen:

\newcommand{\ZitatStil}{apa} % (apa oder ieee)

\ifthenelse{\equal{\ZitatStil}{apa}}{%
  \usepackage[sortlocale=auto,sorting=nyt,style=apa]{biblatex}%
}{
\ifthenelse{\equal{\ZitatStil}{iee}}{%
  \usepackage[backend=biber,style=ieee,dashed=false,]{biblatex}%
}}


% Verwalten von Abkuerzungen und einem Abkuerzungsverzeichnis
\usepackage[printonlyused]{acronym} % https://www.ctan.org/pkg/acronym

% Packages for forcing floats https://robjhyndman.com/hyndsight/latex-floats/
\usepackage{afterpage}
\usepackage[section]{placeins}
\usepackage{censor}


% Erstellen eines Symbolverzeichnisses
\usepackage[intoc, english, stdsubgroups]{nomencl} % https://www.ctan.org/pkg/nomencl
\usepackage{siunitx}
\newcommand{\nomunit}[1]{%
    \renewcommand{\nomentryend}{\hspace*{\fill}\si{#1}}}
\makenomenclature

%TODO Definition der Darstellung von Source Code
\usepackage[newfloat, cache]{minted} % https://www.ctan.org/pkg/minted
\SetupFloatingEnvironment{listing}{name=Listing, placement=b}
\SetupFloatingEnvironment{listing}{listname={Listingverzeichnis}}
\setminted[java]{linenos, fontsize=\footnotesize, frame=lines, breaklines, breakbefore={.}}
\setminted[python]{linenos, fontsize=\footnotesize, frame=lines, breaklines, breakbefore={.}}
\setminted[js]{linenos, fontsize=\footnotesize, frame=lines, breaklines, breakbefore={.}}
\setminted[json]{linenos, fontsize=\footnotesize, frame=lines, breaklines, breakbefore={.}}
\setminted[text]{fontsize=\footnotesize, frame=lines}
\usemintedstyle{friendly}
% new environment for listings longer than one page; https://tex.stackexchange.com/a/53540
\newenvironment{longlisting}{\captionsetup{type=listing}}{}
\usepackage{csquotes}


% Darstellung von Algorithmen und Pseudocode
% do NOT use option naturalnames if you compile with pdflatex and use hyperref
\usepackage[linesnumbered, commentsnumbered, ruled]{algorithm2e} 
\renewcommand{\listalgorithmcfname}{Algorithmusverzeichnis}
\addtotoclist[float]{loa}
\renewcommand\listofalgorithms{\listoftoc[{\listalgorithmcfname}]{loa}}
%\SetAlFnt{\small \normalfont \sffamily} 
\SetAlFnt{\footnotesize} 


% Erstellung neuer Verzeichnisse; Code weitgehend von Markus Kohm und komascript.de 
\DeclareNewTOC[%
  owner=anhang,
  listname={Appendix Listing},% Titel des Verzeichnisses
]{atoc}

\DeclareNewTOC[%
  type=equation,
  listname={Formelverzeichnis},
  tocentrynumwidth=2.3em,
]{loe}

\makeatletter
\renewcommand\@pnumwidth{2em} % vermeiden von overful hbox im Inhaltsverzeichnis

\AfterTOCHead[atoc]{\let\if@dynlist\if@tocleft}
\newcommand*{\useappendixtocs}{%
  \renewcommand*{\ext@toc}{atoc}%
  \scr@ifundefinedorrelax{hypersetup}{}{% damit es auch ohne hyperref funktioniert
    \hypersetup{bookmarkstype=atoc}%
  }%
}
\newcommand*{\usestandardtocs}{%
  \renewcommand*{\ext@toc}{toc}%
  \scr@ifundefinedorrelax{hypersetup}{}{% damit es auch ohne hyperref funktioniert
    \hypersetup{bookmarkstype=toc}%
  }%
  \renewcommand*{\ext@figure}{lof}%
  \renewcommand*{\ext@table}{lot}%
}
\scr@ifundefinedorrelax{ext@toc}{%
  \newcommand*{\ext@toc}{toc}
  \renewcommand{\addtocentrydefault}[3]{%
    \expandafter\tocbasic@addxcontentsline\expandafter{\ext@toc}{#1}{#2}{#3}%
  }
}{}
\newcommand*{\@currententry}{}
% Zwei amsmath-Anweisungen ändern:
\g@addto@macro\make@display@tag{\set@currententry}%
\def\tagform@#1{\maketag@@@{(\ignorespaces#1\unskip\@@italiccorr)}%
  \set@currententry}
\newcommand*{\set@currententry}{%
  \typeout{set current entry}%
  \ifx\@currententry\@empty\else
    \addcontentsline{loe}{equation}{\protect\numberline{\@currentlabel}%
      \@currententry}%
    \global\let\@currententry\@empty
  \fi
}
% Neue Benutzeranweisung
\newcommand*{\equationentry}[1]{%
  \gdef\@currententry{#1}%
}

\makeatother

\usepackage{xpatch}
\xapptocmd\appendix{%
  \useappendixtocs
  \pdfbookmark{Appendix}{appendix}
  \listofatocs
  \addcontentsline{toc}{chapter}{Appendix}
  \bookmarksetupnext{level=-1}
}{}{}

% Pakete, die fuer Informatik Sinn ergeben koennten
% \usepackage{bytefield} % illustration of fields of data https://www.ctan.org/pkg/bytefield


% Uebersetzung fuer Eintraege im Abkuerzungsverzeichnis - Code uebernommen von https://tex.stackexchange.com/a/135507
\makeatletter
\newcommand{\acroforeign}[1]{}

% patch the environment to print the foreign definition:
\AtBeginEnvironment{acronym}{%
  \def\acroforeign#1{ (#1)}%
}

% patch the acronym definition to safe the foreign definition:
\expandafter\patchcmd\csname AC@\AC@prefix{}@acro\endcsname
  {\begingroup}
  {\begingroup\def\acroforeign##1{\csdef{ac@#1@foreign}{##1, }}}
  {}
  {\fail}

% %   renew the first output to include the foreign definition if given:
\renewcommand*{\@acf}[2][\AC@linebreakpenalty]{%
  \ifAC@footnote
    \acsfont{\csname ac@#2@foreign\endcsname\AC@acs{#2}}%
    \footnote{\AC@placelabel{#2}\AC@acl{#2}{}}%
  \else
    \acffont{%
      \AC@placelabel{#2}\AC@acl{#2}%
      \nolinebreak[#1] %
      \acfsfont{(\acsfont{\csname ac@#2@foreign\endcsname\AC@acs{#2}})}%
    }%
  \fi
  \ifAC@starred\else\AC@logged{#2}\fi
}
\makeatother

% Adjusting the width that is reserved for the  pagenumber in listings
% https://projekte.dante.de/DanteFAQ/Verzeichnisse#2
% https://de.comp.text.tex.narkive.com/fAP3Znev/overfull-hbox-im-inhaltsverzeichnis
\makeatletter
\AtBeginDocument{
\newlength{\mylen}
\setlength{\mylen}{\widthof{XVIII}} %TODO hier Text eintragen, der der breitesten Seitennummer im Inhaltsverzeichnis oder einem der anderne Verzeichnisse entspricht
\renewcommand*\@pnumwidth{\the\mylen}
}
\makeatother

% Allow a chapter to occur without a headline to be able to cram both language abstracts on page
\makeatletter
\newcommand{\unchapter}[1]{%
  \begingroup
  \let\@makechapterhead\@gobble % make \@makechapterhead do nothing
  \chapter{#1}
  \endgroup
}
\makeatother
